\usepackage{xparse}

\newtheoremstyle{lemm}% name
{0pt}%      Space above
{0pt}%      Space below
{\itshape}%         Body font
{}%         Indent amount (empty = no indent, \parindent = para indent)
{\bfseries}% Thm head font
{.}%        Punctuation after thm head
{\newline}%     Space after thm head: " " = normal interword space;
%       \newline = linebreak
{}%         Thm head spec (can be left empty, meaning `normal')


\theoremstyle{lemm}

\newtheorem{lemm}{Lemma}[chapter]
\newtheorem{propo}[lemm]{Proposition}
\newtheorem{problem}[lemm]{Problem}
\newtheorem{theom}[lemm]{Theorem}
\newtheorem{korr}[lemm]{Corollary}
\newtheorem{defin}[lemm]{Definition}
%\newtheorem{bspn}[lemm]{Beispiel}
\newtheoremstyle{example}% name
{4pt}%      Space above
{4pt}%      Space below
{\itshape}%         Body font
{}%         Indent amount (empty = no indent, \parindent = para indent)
{\bfseries}% Thm head font
{.}%        Punctuation after thm head
{ }%     Space after thm head: " " = normal interword space;
%       \newline = linebreak
{}%         Thm head spec (can be left empty, meaning `normal')

\theoremstyle{example}
\newtheorem{examp}[lemm]{Example}
\newtheoremstyle{remark}% name
{4pt}%      Space above
{4pt}%      Space below
{\itshape}%         Body font
{}%         Indent amount (empty = no indent, \parindent = para indent)
{\bfseries}% Thm head font
{.}%        Punctuation after thm head
{ }%     Space after thm head: " " = normal interword space;
%       \newline = linebreak
{}%         Thm head spec (can be left empty, meaning `normal')

\theoremstyle{remark}
\newtheorem{rem}[lemm]{Remark}
\newtheorem*{nota}{Notation}
\newtheorem*{proper}{Properties}


\usepackage{framed}

\makeatletter
\newcommand{\colorprovide}[2]{%
\@ifundefinedcolor{#1}{\colorlet{#1}{#2}}{}}
\makeatother

\colorprovide{thmcolor}{blue!20}
\colorprovide{corcolor}{gray!15}
\colorprovide{deficolor}{orange!20}
\colorprovide{lemcolor}{gray!15}
\colorprovide{propcolor}{blue!15}

\NewDocumentEnvironment{thm}{oo}{\colorlet{shadecolor}{thmcolor}%
    \begin{shaded}\IfNoValueTF{#1}{\begin{theom}}{\begin{theom}[#1]}%
    \IfNoValueTF{#2}{\!}{#2~\!\!\!}}%
{\end{theom}\end{shaded}}


\NewDocumentEnvironment{cor}{oo}{\colorlet{shadecolor}{corcolor}%
    \begin{shaded}\IfNoValueTF{#1}{\begin{korr}}{\begin{korr}[#1]}%
    \IfNoValueTF{#2}{\!}{#2~\!\!\!}}%
{\end{korr}\end{shaded}}

\NewDocumentEnvironment{defi}{oo}{\colorlet{shadecolor}{deficolor}%
    \begin{shaded}\IfNoValueTF{#1}{\begin{defin}}{\begin{defin}[#1]}%
    \IfNoValueTF{#2}{\!}{#2~\!\!\!}}%
{\end{defin}\end{shaded}}

\NewDocumentEnvironment{lem}{oo}{\colorlet{shadecolor}{lemcolor}%
    \begin{shaded}\IfNoValueTF{#1}{\begin{lemm}}{\begin{lemm}[#1]}%
    \IfNoValueTF{#2}{\!}{#2~\!\!\!}}%
{\end{lemm}\end{shaded}}

\NewDocumentEnvironment{prop}{oo}{\colorlet{shadecolor}{propcolor}%
    \begin{shaded}\IfNoValueTF{#1}{\begin{propo}}{\begin{propo}[#1]}%
    \IfNoValueTF{#2}{\!}{#2~\!\!\!}}%
{\end{propo}\end{shaded}}

\usepackage{algorithmicx}
\usepackage{algpseudocode}
\usepackage{algorithm}
\newcommand*\Let[2]{\State $ #1 \gets #2 $}

\renewcommand{\algorithmicrequire}{\textbf{Input:}}
\renewcommand{\algorithmicensure}{\textbf{Output:}}
